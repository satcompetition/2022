\documentclass{article}

\bibliographystyle{plainurl}

\usepackage[T1]{fontenc} 
\usepackage[utf8]{inputenc}
\usepackage{graphicx}
\usepackage{amsmath}
\usepackage{amssymb}
\usepackage[ruled,linesnumbered,vlined]{algorithm2e}
\usepackage{xcolor}
\usepackage{tabularx}
\usepackage{tabu}
\usepackage{todonotes}
\usepackage{tabto}
\usepackage{url}
\usepackage{hyperref}
\usepackage{caption} 
%\usepackage[left=4cm, right=4cm]{geometry}


\title{Call for Verified Proof Checkers}
\author{Please send your proposal to\\[.3em] \mbox{\url{organizers@satcompetition.org}}\\[.3em] by 31st~of~December~2022.}
\date{}

\begin{document}
\maketitle
\thispagestyle{empty}


We seek proposals of new proof checkers for the international SAT Competitions. 
Your proposal should specify a proof format and include an actual proof checker implementation for that format.
Please provide an empirical evaluation of the \emph{performance} of your proof checker implementation.

Your proposal should address the \emph{complexity} of the proposed system. 
This includes an overview of its relationship to other popular proof systems such as DRAT. %~\cite{Wetzler:2014:DRATtrim}. 
Clearly specify the benefits and limitations of the proposed proof system in terms of worst-case lower bounds on proof lengths. 

The proposed checker implementation must be \emph{formally verified}. 
In your proposal, you should elaborate on the properties that were checked and the tools or methods used to do so.
References to previous publications that support your statements are helpful. 


%\subsubsection*{Proof-Complexity and Performance}
%\label{sec:perf}
%The evaluation should contain a comparison to \texttt{DRAT-trim}~\cite{Wetzler:2014:DRATtrim}~\cite{Tan:2021:CakeLPR}.
%Briefly outline the underlying proof system which is used by the proposed checker. 


%\subsubsection*{Compatibility and Genericity}
%\label{sec:comp}
% and on how well it \emph{integrates} into existing infrastructure and systems.
%The proof format should be easy to adopt by future SAT solvers.
%You should report on how well the proposed checker integrates into the existing pre-dominant infrastructures. 
%A statement on the compatibility of the proposed proof format should encompass if it subsumes DRAT and how much it costs to translate between the two.
%You should also elaborate on the genericity of the proposed proof system, i.e., the extent to which the proposed checker can be used in domains other than SAT. 

%\bibliography{cfvpc}


\end{document}




